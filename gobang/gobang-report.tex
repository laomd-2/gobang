\documentclass[UTF8,a4paper,12pt]{article}

\usepackage[version=3]{mhchem} % Package for chemical equation typesetting
\usepackage{ctex}
\usepackage{siunitx} % Provides the \SI{}{} and \si{} command for typesetting SI units
\usepackage{graphicx} % Required for the inclusion of images
\usepackage{natbib} % Required to change bibliography style to APA
\usepackage{amsmath} % Required for some math elements
\usepackage{enumitem}
\usepackage{indentfirst}

\usepackage[top=2cm, bottom=2cm, left=2cm, right=2cm]{geometry}
\usepackage{algorithm}
\usepackage{algorithmicx}
\usepackage{algpseudocode}

\renewcommand{\labelenumi}{\alph{enumi}.} % Make numbering in the enumerate environment by letter rather than number (e.g. section 6)
\floatname{algorithm}{算法}
\renewcommand{\algorithmicrequire}{\textbf{输入:}}
\renewcommand{\algorithmicensure}{\textbf{输出:}}

\usepackage{listings}
% \usepackage{times} % Uncomment to use the Times New Roman font
%----------------------------------------------------------------------------------------
%	DOCUMENT INFORMATION
%----------------------------------------------------------------------------------------

\begin{document}

\begin{titlepage}
    \begin{center}
        \phantom{Start!}
    	  \vspace{2cm}
        \center{\zihao{1} 中山大学数据科学与计算机学院}
        \center{\zihao{1} 程序设计与数据结构综合实践II}
        \center{\zihao{1} “五子棋”实验报告}
        \center{(2018-2019学年秋季学期)}
        {
            \setlength{\baselineskip}{40pt}
            \vspace{1cm}
            \zihao{-2}
            \center{
                \begin{tabular}{cc}
              	学\qquad 号:& \underline{~~~~~~16337113~~~~~~}  \\
              	姓\qquad 名:& \underline{~~~~~~~劳马东~~~~~~~}  \\
                教学班级:   & \underline{~~~~~教务2班~~~~~}  \\
              	专\qquad 业:& \underline{~~~~~~~~~超算~~~~~~~~}  \\
              	\end{tabular}
            }
        }
    \end{center}
\end{titlepage}

\section{实验题目}
\par 实现一个双人五子棋程序,每一次下棋操作都会输出棋谱chessboard.txt,
0表示未下子的位置,1表示先手(即黑棋下的位置),2表示后手(即白棋下的位置)。
chessboard.txt作为ai程序的输入,用来得到ai下一步将会下在什么位置,然后下在棋盘对应位置。
棋盘大小为15×15,有禁手规则。
% \begin{enumerate}[itemindent=0.5em,label=\arabic*、]
% \end{enumerate}
\section{实验设计}
\subsection{游戏主循环}
\begin{figure}[h]
\begin{center}
\includegraphics[width=0.8\textwidth]{p1.png} % Include the image placeholder.png
\caption{游戏主循环}
\end{center}
\end{figure}
\subsection{类设计}
\begin{figure}[h]
\begin{center}
\includegraphics[width=0.8\textwidth]{p2.png} % Include the image placeholder.png
\caption{主要类关系}
\end{center}
\end{figure}
\begin{enumerate}[itemindent=0.5em,label=\arabic*、]
  \item ChessBoard类
  \begin{enumerate}[itemindent=0.5em,label=(\arabic*)]
    \item 成员变量$chess\_board$是一个二维数组(初始化为全0),保存棋盘的落子信息。
    \item $last\_drop$保存上一次落子的位置,用于在输出棋盘时高亮输出落子位置。
    \item $get$和$drop$操作分别获取和设置棋盘特定坐标处的值。
    \item $check\_winner$利用$have\_five$函数检查是否有五个同色棋子,即是否有一方胜出。
    \item $count\_on\_direction$函数统计以$origin$为原点,$direction$向量方向同色的棋子数目。
  \end{enumerate}
  \item AI类
  \begin{enumerate}[itemindent=0.5em,label=(\arabic*)]
    \item $chess\_board$成员变量是一个ChessBoard实例,用于决定落子位置。
    \item $decide\_next\_pos$函数利用$evaluate$函数计算每个空余位置的评分,并返回对于$turn$方来说最优的落子位置。
    \item $evaluate$函数根据某种原则计算假如在$pos$位置落子对$turn$方有多大好处。
  \end{enumerate}
\end{enumerate}
\subsection{关键代码}
\begin{enumerate}[itemindent=0.5em,label=\arabic*、]
  \item 决定落子位置的估分函数\\
  棋盘的每个位置有三种状态,假设分别为0(己方落子),1(空),2(敌方落子)。那么,当要找最优
  落子位置时,需检验在空位置落子后以该位置为起点的射线上的五个点的状态。显然,起点的状态是固定的(当前方的棋子颜色),
  其余四个位置的状态有如下$3 \times 3 \times 3 \times 3 = 81$种9类,给每类情况赋予合理的分数,在遍历时累加,就能找出得分最高的落子位置。
  \begin{center}
    \begin{tabular}{cccccc}
    \hline
    预落子 & 1 & 2 & 3 & 4 & 取值个数\\
    \hline
    0 & 0 & 0 & 0 & 0 & 1\\
    0 & 0 & 0 & 0 & 1 & 1\\
    0 & 0 & 0 & 0 & 2 & 1\\
    0 & 0 & 0 & 1 & * & 3\\
    0 & 0 & 0 & 2 & * & 3\\
    0 & 0 & 1 & * & * & 9\\
    0 & 0 & 2 & * & * & 9\\
    0 & 1 & * & * & * & 27\\
    0 & 2 & * & * & * & 27\\
    \hline
    \end{tabular}
  \end{center}
  \begin{figure}[h]
  \begin{center}
  \includegraphics[width=0.8\textwidth]{p4.png} % Include the image placeholder.png
  \caption{分数设定}
  \end{center}
  \end{figure}
  \begin{figure}[h]
  \begin{center}
  \includegraphics[width=0.8\textwidth]{p3.png} % Include the image placeholder.png
  \caption{估分函数}
  \end{center}
  \end{figure}
\end{enumerate}
% \begin{algorithm}
%       \begin{algorithmic}[1] %每行显示行号
%           \Require 训练集X,训练集y,父节点$parent$,分支的取值$brach$
%           \Function {$build\_tree$}{$X, y, parent, brach$}
              % \State $F \gets feature(X)$
              % \If {$X \ is \  empty$}
              %   \State $root.label \gets parent.label$
              % \ElsIf {$is\_same(y)$}
              %   \State $root.label \gets y[0]$
              % \Else
              % \For {each ${F_j} \in F$}
              % \EndFor
              % \EndIf
              % \State \Return{$root$}
  %         \EndFunction
  %     \end{algorithmic}
  % \end{algorithm}


\pagebreak
\section{实验结果}
图中1表示黑棋(玩家,先手),2表示白棋(电脑),0表示未落子,黄色字体表示上一次落子。从图中可以
看出白棋在第四列连成5个,胜出,程序打印“White win”并退出。
\begin{figure}[h]
\begin{center}
\includegraphics[width=0.342\textwidth]{p5.png} % Include the image placeholder.png
\caption{估分函数}
\end{center}
\end{figure}
\end{document}
